\begin{center}
\textbf{\huge Personal Emergency Response}
\end{center}

\begin{itemize}
\item \textbf{Food}

Keep one week worth of groceries on hand at all times, along with food for emergencies (granola bars, etc.)

Keep at least a 3-day supply of emergency water (one gallon per person per day).

\item \textbf{Transportation}

Keep your car’s gas tank at least half full at all times.

\item \textbf{Money}

Have enough money available to get to your parent’s home (if that is where you would go in case of an emergency).

\item \textbf{Communication}

Have your cell phone programmed to call family and other important people in your life. Program an ICE (In Case of Emergency) number in your cell phone directory.

Designate an out of area family member as a family communication contact.

Keep your roommates/spouse apprised of your whereabouts.

Know about emergency info sources, including KSL AM1160 and FM 102.7 and KBYU FM 89.1 and 89.5. Remember that your car radio is a source for emergency info.

\item \textbf{Other Emergency Items}

Designate a place for meeting your roommates, or your spouse and children (right outside your home for emergencies such as fires, and outside your neighborhood if you can’t get home).

Be aware of your ward’s emergency response plan, especially the ward emergency locations.

Identify primary and alternate escape routes out of your home and conduct drills with your family/roommates.

Keep all needed medications readily available (one-week supply).

Have items available for warmth in cold weather (coats, blankets, etc)

Keep insurance policies (policy \# and contact information) available, along with any other important documents, such as birth certificates and marriage licenses.

Learn what to do for the different hazards that could impact you or your family.

Go to http://risk.byu.edu/emergency for more information
\end{itemize}
\textbf{The Provo YSA 32nd Ward emergency meeting place is the space in between the Thomas L. Martin Building (MARB) and Life Science Building (LSB).}